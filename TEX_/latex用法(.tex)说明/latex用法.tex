\documentclass[UTF8]{ctexart}

% 英文的开头可以 \documentclass{article}

% 套了一个学术论文模板后大致了解latex的笔记

% 导言区    通常在导言区设置页面大小、页眉页脚样式、章节标题样式等等。
%数学公式插入
\usepackage{amsmath}
%链接插入
\usepackage{url}
\usepackage[colorlinks,linkcolor=red]{hyperref}
%图片插入
\usepackage{graphicx}
%页边距 比如我希望,将纸张的长度设置为 20cm、宽度设置为 15cm、左边距 1cm、右边距 2cm、上边距 3cm、下边距 4cm,可以在导言区加上这样几行:
%\usepackage{geometry}
%\geometry{papersize={20cm,15cm}}
%\geometry{left=1cm,right=2cm,top=3cm,bottom=4cm}
%页眉页脚 比如我希望,在页眉左边写上我的名字,中间写上今天的日期,右边写上我的电话;页脚的正中写上页码;页眉和正文之间有一道宽为 0.4pt 的横线分割,可以在导言区加上如下几行:

\usepackage{fancyhdr}
\pagestyle{fancy}
\lhead{\author}
\chead{\date}
\rhead{152xxxxxxxx}
\lfoot{}
\cfoot{\thepage}
\rfoot{}
\renewcommand{\headrulewidth}{0.4pt}
\renewcommand{\headwidth}{\textwidth}
\renewcommand{\footrulewidth}{0pt}


\title{latex简单使用}
\author{魏卓}
\date{\today}

\begin{document}
	\maketitle
	\tableofcontents  % 插入目录
	% 控制序列 maketitle,这个控制序列能将在导言区中定义的标题、作者、日期按照预定的格式展现出来。
	
	\href{https://liam.page/2014/09/08/latex-introduction/}{本文参考}
	\section{latex说明,安装}
	LaTeX —— 这个优雅,但有着自己高傲,却绝不复杂甚至神秘的东西
	
	Tex安装 下载Texlive(发行的Tex,另一个是MiKTex) + Texstudio 随便哪篇博客都有,设置上没啥大问题 
	
	Latex套模板、必须在模板环境(即模板的路径里)里运行。
	
	TeXworks 是 TeX Live 自带的编辑器,而 TeX Live 是 TeX User Group 出品的跨平台发行版,各个操作系统都可以使用
	
	% 所有控制序列用 \开头
	
	这里的控制序列是 documentclass,它后面紧跟着的 {article} 代表这个控制序列有一个必要的参数,该参数的值为 article。这个控制序列的作用,是调用名为 article 的文档类。
	
	% 注释
	
	今年的净利润为 20\%,比去年高。 %加上\ 保证“%” 不被注释
	
	\subsection{宏包}
	\paragraph{所谓宏包,就是一系列控制序列的合集}
	\subparagraph{所谓宏包,}就是一系列控制序列的合集。这些控制序列太常用,以至于人们会觉得每次将他们写在导言区太过繁琐,于是将他们打包放在同一个文件中,成为所谓的宏包(台湾方面称之为「巨集套件」)。
	%   \usepackage{} 可以用来调用宏包
	CTeX 宏集一次性解决了这些问题。CTeX 宏集的优势在于,它能适配于多种编译方式;在内部处理好了中文和中文版式的支持,隐藏了这些细节;并且,提供了不少中文用户需要的功能接口。
	
	
	\subsubsection{字体}
	按下% <win> + R;  
	键入cmd, 回车。
	在系统命令行中输入如下命令:
	
	% fc-list :lang=zh-cn > C:\font_zh-cn.txt
	用vsc打开
	其中的每一行,都代表着一个可用的字体。其形式如下:
	
	<字体文件路径>: <字体标示名1>, <字体表示名2>:Style=<字体类型>
	我们可以看到图中的倒数第四行
	
	% C:/WINDOWS/fonts/simsun.ttc: 宋体,SimSun:style=Regular
	% 出现了之前文档里调用的字体SimSun,此处表明该字体有两个表示名:宋体和SimSun,我们在\setCJKmainfont{·}中填入任意一个都有同样的效果。
	
	\section{数学公式}
	% 需要导言部分用 \usepackage{amsmath}
	
	% 在行文中,使用 $ ... $ 可以插入行内公式,使用 \[ ... \] 可以插入行间公式,如果需要对行间公式进行编号,则可以使用 equation 环境:
	
	%\begin{equation}
	%    ...
	%\end{equation}
	Einstein 's $E=mc^2$.
	
	\[ E=mc^2. \]
	
	\begin{equation}
	E=mc^2.
	\end{equation}
	
	\[ z = r\cdot e^{2\pi i}. \]
	根式用 $\sqrt{x} $来表示,分式用 $\frac{a}{b} $来表示(第一个参数为分子,第二个为分母)。
	
	一些小的运算符,可以在数学模式下直接输入;另一些需要用控制序列生成,如
	
	\[ \pm\; \times\; \div\; \cdot\; \cap\; \cup\; \geq\; \leq\; \neq\; \approx \; \equiv \]
	
	连加、连乘、极限、积分等大型运算符分别用 % \sum, \prod, \lim, \int 
	生成。他们的上下标在行内公式中被压缩,以适应行高。我们可以用 %\limits 和 \nolimits
	来强制显式地指定是否压缩这些上下标。例如:
	
	$ \sum_{i=1}^n i\quad \prod_{i=1}^n  \sum\limits _{i=1}^n i\quad \prod\limits _{i=1}^n $
	\[ \lim_{x\to0}x^2 \quad \int_a^b x^2 dx \]
	\[ \lim\nolimits _{x\to0}x^2\quad \int\nolimits_a^b x^2 dx \]
	
	多重积分可以使用$ \iint, \iiint, \iiiint, \idotsint $等命令输入。
	
	\[ \iint\quad \iiint\quad \iiiint\quad \idotsint \]
	
	\subsection{括号}
	各种括号用 (), [], \{\},% \langle\rangle
	等命令表示;注意花括号通常用来输入命令和环境的参数,所以在数学公式中它们前面要加 % \。
	
	更复杂的参见 \href{https://liam.page/2018/11/09/the-bigger-than-bigger-delimiter-in-LaTeX/}{复杂公式}
	顺便写一下邮箱引用	\href{mailto: 1327905168@qq.com}{邮箱}
	
	\[ \Biggl(\biggl(\Bigl(\bigl((x)\bigr)\Bigr)\biggr)\Biggr) \]
	
	\[ \Biggl[\biggl[\Bigl[\bigl[[x]\bigr]\Bigr]\biggr]\Biggr] \]
	
	\[ \Biggl \{\biggl \{\Bigl \{\bigl \{\{x\}\bigr \}\Bigr \}\biggr \}\Biggr\} \]
	
	\[ \Biggl\langle\biggl\langle\Bigl\langle\bigl\langle\langle x
	\rangle\bigr\rangle\Bigr\rangle\biggr\rangle\Biggr\rangle \]
	
	\[ \Biggl\lvert\biggl\lvert\Bigl\lvert\bigl\lvert\lvert x
	\rvert\bigr\rvert\Bigr\rvert\biggr\rvert\Biggr\rvert \]
	
	\[ \Biggl\lVert\biggl\lVert\Bigl\lVert\bigl\lVert\lVert x
	\rVert\bigr\rVert\Bigr\rVert\biggr\rVert\Biggr\rVert \]
	
	省略号用
	% \dots 和 \cdots 的纵向位置不同,前者一般用于有下标的序列。
	
	\[ x_1,x_2,\dots ,x_n\quad 1,2,\cdots ,n\quad \vdots\quad \ddots \]
	
	\subsection{矩阵}
	凭网上公式生成写吧
	
	\subsection{公式组}
	
	\begin{gather}
	a = b+c+d \\
	x = y+z
	\end{gather}
	
	\begin{align}
	a &= b+c+d \\
	x &= y+z
	\end{align}
	
	
	请注意,不要使用 eqnarray 环境
	
	\paragraph{分段函数}
	\[ y= \begin{cases}
	-x,\quad x\leq 0 \\
	x,\quad x>0
	\end{cases} \]
	
	\subsection{辅助工具}
	
	\href{https://mathpix.com/ }{能够通过热键呼出截屏,而后将截屏中的公式转换成 LaTeX 数学公式的代码}
	
	\href{http://detexify.kirelabs.org/classify.html}{允许用户用鼠标在输入区绘制单个数学符号的样式,系统会根据样式返回对应的 LaTeX 代码}
	
	
	\section{图片表格}
	\subsection{图片}
	在 LaTeX 中插入图片,有很多种方式。最好用的应当属利用 graphicx 宏包提供的 % \includegraphics 
	命令。比如你在你的 TeX 源文件同目录下,有名为 a.jpg 的图片,你可以用这样的方式将它插入到输出文档中:
	%\includegraphics[width = .8\textwidth]{a.jpg}   假设同目录下有a.jpg时
	
	\paragraph{浮动体}
	插图和表格通常需要占据大块空间,所以在文字处理软件中我们经常需要调整他们的位置。figure 和 table 环境可以自动完成这样的任务;这种自动调整位置的环境称作浮动体(float)。我们以 figure 为例。
	
	%\begin{figure}[htbp]
	%\centering
	%\includegraphics{a.jpg}
	%\caption{有图有真相}
	%\label{fig:myphoto}
	%\end{figure}
	
	htbp 选项用来指定插图的理想位置,这几个字母分别代表 here, top, bottom, float page,也就是就这里、页顶、页尾、浮动页(专门放浮动体的单独页面或分栏)。
	% \centering 用来使插图居中;\caption 命令设置插图标题,LaTeX 会自动给浮动体的标题加上编号。注意 \label 应该放在标题命令之后。
	
	\subsection{表格}
	使用辅助工具,百度搜latex表格
	
	
	\section{其他说明}
	LaTeX 将一个换行当做是一个简单的空格来处理,如果需要换行另起一段,则需要用两个换行(一个空行)来实现。
	数模模板 
	\href{https://liam.page/2016/01/27/how-to-use-mcmthesis/}{如何使用美赛模板 mcmthesis}
	
	
\end{document}